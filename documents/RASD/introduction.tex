\chapter{Introduction}

\section{Purpose}

Students\&Companies (S\&C) is a platform designed to connect university students with companies offering internships.
It simplifies the internship searches of students and the projects advertisement for companies.

The platform employs recommendation mechanisms to match students and companies based on experience, skills, and project requirements.
S\&C also supports the selection process by managing interviews and collecting feedbacks.
Additionally, it provides suggestions for improving CVs and project descriptions.

\subsection{Goals}

\begin{enumerate}[label=\textbf{G\arabic* -}]
   \item Allow registered students to search and enroll for internship opportunities.
   \item Allow registered companies to advertise internship project opportunities.
   \item Allow registered universities to monitor their students ongoing internship and manage complaints.
   \item Support companies in the selection process by providing students with custom-made questionnaires.
   \item Ease matching by notifying students of relevant internships and companies for suitable candidates.
   \item Provide suggestions to both parties to refine their submissions.
\end{enumerate}

\section{Scope}

\subsection{Features}

\begin{enumerate}[label=\textbf{F\arabic* -}]
    \item \textbf{Recommendation mechanism}: students and companies are matched based on skills, experience and project requirements, exploiting keyword statistical methods.
    \item \textbf{Internship search}: students actively search and enroll for internships, while also being notified of opportunities that align with their profile.
    \item \textbf{Selection support}: companies are helped in the selection process by setting up interviews, creating custom questionnaires, and finalizing selections.
    \item \textbf{Suggestions system}: suggestions are provided to both parties in order to improve CVs and project descriptions, to enhance match potential.
    \item \textbf{Complaint management}: universities can monitor internships of their students, handling complaints and addressing issues.
\end{enumerate}

\subsection{World phenomena}

\begin{enumerate}[label=\textbf{WP\arabic* -}]
    \item A user signs up in the system.
    \item A user logs in the system.
    \item A student looks for an internship to which enroll.
    \item A company advertises an open internship project.
    \item A student candidates himself for an internship.
    \item A company asks a matched student to fill a questionnaire.
    \item A company accepts a student for its internship project.
    \item A student wants to visualize its ongoing internship information.
    \item A company wants to visualize its ongoing internships information.
    \item A university wants to visualize its student ongoing internship information.
    \item A student or a company sends a compaint to the university.
    \item A student or a company fills a feedback form when internship ends.
\end{enumerate}

\subsection{Shared phenomena}

\begin{enumerate}[label=\textbf{SP\arabic* -}]

\subsubsection{Controlled by the machine}

    \item The system shows the sign up page to a user.
    \item The system shows the log in page to a user.
    \item The system shows some internship advertisements to the student.
    \item The system shows some eligible student information to the company.
    \item The system notifies companies about students enrollment requests.
    \item The system notifies students about companies candidation offers.
    \item The system shows the company a panel in which to create a custom questionnaire.
    \item The system shows the student the questionnaire to fill.
    \item The system assigns the student to an internship project of a company.
    \item The system shows the student its ongoing internship information.
    \item The system shows the company its ongoing internships information.
    \item The system notifies a student or a company with a suggestion.
    \item The system notifies the university of a complaint.
    \item The system ends the internship of a student.
    \item The system shows a participant the feedback form to fill in.

\subsubsection{Controlled by the world}

    \item A user fills the registration form to sign up.
    \item A user fills the credentials fields to log in.
    \item A student fills the form with its personal information and uploads its CV.
    \item A company fills the form for advertising an internship project.
    \item A student sends an enrollment request for an internship project.
    \item A company sends an enrollment suggestion to a student.
    \item A company creates a custom questionnaire.
    \item A company sends a questionnaire to a student.
    \item A student fills a questionnaire a company sent.
    \item A company accepts a candidate student for its internship project.
    \item A student or a company sends a complaint to the university.
    \item A university ends the internship of its student.
    \item A participant fills a feedback form.

\end{enumerate}

\section{Definitions, acronyms, abbreviations}

\subsection{Definitions}

\begin{itemize}
    \item \textbf{Internship project}: the description of the skills, technologies and roles the student will be working with during the internship, along with the set of tasks that will be covered
    \item \textbf{Internship advertisement}: the public post created by companies to promote available internships on the platform, aimed at attracting suitable candidates by highlighting its key aspects
    \item \textbf{Internship information}: general data about the (ongoing) internship, including the elapsed and remaining time, the compensation and the description of the project the student is working on
    \item \textbf{Enrollment request}: the submission of a student to indicate interest in a specific internship, initiating the selection process by formally applying
    \item \textbf{Enrollment suggestion}: the recommendation made by the platform to guide students in finding projects that best suit them
    \item \textbf{Custom questionnaire}: the tailored set of questions used by companies during interviews to assess a candidate fit for the internship
    \item \textbf{Candidate student}: a student who has applied for an internship and is currently under consideration by a company, moving forward in the selection process
    \item \textbf{Eligible student}: a student who meets the qualifications for an internship, making them viable candidates for recommendation and application
    \item \textbf{Suitable student}: a student who meets the qualifications for an internship, making them potential candidates to be recommended in the companies feed
    \item \textbf{Complaint}: a report submitted by a student or company to the university, regarding issues during the internship, such as unmet expectations, mistreatments, or procedural problems
    \item \textbf{Feedback form}: a structured form for students and companies used to provide feedback on their internship experience, enabling the platform to gather data for analysis, improvements, and recommendations
\end{itemize}

\subsection{Acronyms}

\begin{itemize}
    \item \textbf{S\&C}: Students\&Companies
\end{itemize}

\subsection{Abbreviations}

\begin{itemize}
    \item \textbf{Gn}: n-th goal
    \item \textbf{Fn}: n-th feature
    \item \textbf{WPn}: n-th world phenomena
    \item \textbf{SPn}: n-th shared phenomena
    \item \textbf{Sn}: n-th scenario
    \item \textbf{KFn}: n-th key function
    \item \textbf{Rn}: n-th requirement
    \item \textbf{Dn}: n-th domain assumption
    \item \textbf{UCn}: n-th use case
\end{itemize}

\section{Revision history}

\begin{itemize}
    \item \textbf{Revised on}: \today
    \item \textbf{Version}: 1.0
    \item \textbf{Description}: document initial release
\end{itemize}

\section{Reference documents}

\begin{itemize}
    \item \textbf{Polimi Software Engineering 2 AY 2024/2025 assignment document}: goal, schedule and rules of the requirement engineering and design project
    \item \textbf{Polimi Software Engineering 2 AY 2024/2025 course slides}: the lecture slides provided during the course
\end{itemize}

\section{Document structure}

\begin{itemize}
    \item \textbf{Chapter 1}: here is presented the problem statement and an outlining of the system objectives; in the scope subsection, insights into the various world and shared phenomena explain what the system addresses; here are also provided the essential resources for the readers, including definitions and abbreviations, to facilitate a comprehensive understanding of the document.
    \item \textbf{Chapter 2}: a comprehensive overview of the system is offered including insights into user profiles and their primary functions; the domain diagrams illustrate the system components and describe the various scenarios; the key domain assumptions are established, underpinning the system operations.
    \item \textbf{Chapter 3}: system requirements are delineated, encompassing both functional and non-functional aspects; follows the presentation of use case diagrams, illustrating the system interactions accompanied by their descriptions, with the related sequence diagrams; a clear mapping of the requirements is established, for a comprehensive understanding of both system goals and use cases.
    \item \textbf{Chapter 4}: here is given a formal analysis of the system with Alloy.
    \item \textbf{Chapter 5}: here is found an estimation of the effort spent by each group member.
    \item \textbf{Chapter 6}: here is provided a list of the references used in this document.
\end{itemize}
