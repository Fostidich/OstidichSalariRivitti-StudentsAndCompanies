\chapter{Overall description}

\section{Product perspective}

\subsection{Scenarios}

\begin{enumerate}[label=\textbf{S\arabic* -}]
    \item \textbf{Signing up and logging in}: 
    User Mark opens the platform and starts the sign-up procedure. 
    He fills in the required information and completes the sign-up process. 
    Then he logs out to make sure everything went smoothly, and a few seconds later, he logs in again without any issues.
    \item \textbf {Filling in personal information}: 
    Maria, a newly registered student, logs into the platform and accesses the profile section. 
    She fills in the required fields, such as first name, last name, degree program, and contact details. 
    After completing all the necessary fields, she selects “Save,” and the system confirms that her information has been successfully updated. 
    Now Maria can start applying for internships.
    \item \textbf {Uploading the CV}: 
    After completing their profile information, Luca decides to upload his CV to be ready to apply for internship opportunities. 
    He goes to the "Upload CV" section, selects the PDF file from his computer, and clicks "Upload." 
    The system verifies the file and confirms the upload. 
    Luca now has his CV associated with his profile, ready to be included with applications.
    \item \textbf {Creating an internship project advertisement}: 
    A consulting firm logs into the platform to publish a new internship advertisement. 
    Anna, the recruitment officer, fills in the required fields in the creation form, including the project description, required skills, and internship duration. 
    After verifying the details, Anna selects “Publish.” The system confirms the publication, and the advertisement immediately becomes visible to students on the platform.
    \item \textbf {Notifying the availability of an internship}: 
    Sara has set her preferences to receive notifications about internship opportunities in marketing. 
    When a new company posts a relevant advertisement, the system automatically emails her notification. 
    By opening the email, Sara can quickly access the advertisement on the platform and decide whether to apply.
    \item \textbf {Selecting an internship project}: 
    Alessandro is looking for an internship experience in computer engineering. 
    He accesses the internship opportunities section and browses the available advertisements. 
    When he finds an interesting offer at a startup, he views the details and decides to apply. 
    The system confirms the submission of the application, and Alessandro receives a notification that the application was successfully sent.
    \item \textbf {Creating a custom questionnaire}: 
    A technology company decides to filter candidates with a custom questionnaire on programming skills. 
    Laura, the person in charge, accesses the platform, selects the option to create a questionnaire, and adds specific questions about algorithms and programming languages. 
    After completing the questionnaire, Laura associates it with the internship advertisement, ready to be completed by applicants.
    \item \textbf {Filling a questionnaire}: 
    Martina is applying for an internship as a graphic designer at an advertising agency. 
    During the application process, she is asked to complete a questionnaire evaluating her skills in graphic software like Photoshop and Illustrator. 
    Martina answers all the questions and submits the questionnaire. 
    The system confirms the submission, and Martina has completed her application.
    \item \textbf {Starting a new internship}: 
    Giovanni successfully completes the selection process with the company. 
    Once he has been confirmed, the system notifies both parties and updates the status to "Internship Started." 
    Giovanni can now access the platform to monitor his progress during the internship.
    \item \textbf {Viewing internship information}: 
    During her internship, Laura accesses the "Internship Information" section to monitor project details, including goals, deadlines, and her supervisor's name. 
    The platform allows her to view updates and track her progress, ensuring she meets the requirements and expectations of the internship.
    \item \textbf {Sending a complaint}: 
    Paolo, a student on an internship, encounters issues with the support provided by the company. 
    He decides to report the problem to the university through the platform. 
    He accesses the complaints section, describes the issue, and submits the report. 
    The system notifies the university administration, which reviews the case. 
    After evaluating the situation, the university decides on the appropriate action, such as contacting the company for clarification or providing direct support to Paolo.
    \item \textbf {Ending an internship}: 
    After completing the internship, both the student and the company provide feedback on their experience. 
    The student evaluates the company regarding working conditions, support received, and the relevance of the assigned activities, while the company provides feedback on the skills and performance demonstrated by the student. 
    These feedback reports are collected by the platform and made available to the university for an overall evaluation of the internship experience.
    \item \textbf {Receiving feedback from the company and updating student preparation statistics}:
    At the end of an internship, the university receives feedback from the company evaluating the student's preparation and skills demonstrated during the experience. 
    Upon receiving the feedback, the university updates its student preparation statistics based on the skills demonstrated by the interns. 
    These statistics help the university assess the effectiveness of its academic programs and make improvements as needed
    \item \textbf {Receiving feedback from the student and updating internship program quality}:
    At the end of the internship, the company receives feedback from the student assessing the overall experience, including working conditions, support received, and the relevance of assigned tasks. 
    After receiving the feedback, the company reviews the information to assess the quality of its internship program and make improvements, if necessary, for future internships.
\end{enumerate}

\section{Product functions}

\subsection{Key functions}

\begin{enumerate}[label=\textbf{KF\arabic* -}]
    \item Internship project advertisements
    \item CV upload
    \item Interviews via custom questionnaires
    \item Complaints management
    \item Recommendation system
    \item Suggestion system
\end{enumerate}

\subsection{Requirements}

\begin{enumerate}[label=\textbf{R\arabic* -}]
    \item The system must allow an unregistered student to sign up.
    \item The system must allow an unregistered company to sign up.
    \item The system must allow an unregistered university to sign up.
    \item The system must allow a registered user to log in.
    \item The system must allow a registered user to fill in and edit its personal information.
    \item The system must allow a registered student to upload its CV.
    \item The system must allow a registered company to post an internship project.
    \item The system must allow a registered student to visualize a list of open internship projects.
    \item The system must allow a registered company to visualize a list of eligible students.
    \item The system must allow a registered student to make an enrollment request to an internship project.
    \item The system must allow a registered company to build custom made questionnaires.
    \item The system must allow a registered company to send questionnaires to students.
    \item The system must allow a registered student to fill in the questionnaire.
    \item The system must allow a registered company to accept students enrollment requests.
    \item The system must allow a registered student to see their ongoing internship information.
    \item The system must allow a registered company to see their ongoing internships information.
    \item The system must allow a registered university to see their students ongoing internship information.
    \item The system must allow a registered student to send complaints to the university.
    \item The system must allow a registered company to send complaints to the university.
    \item The system must allow a registered university to visualize complaints it received.
    \item The system must allow a registered university to end an ongoing internship of its student.
\end{enumerate}

\section{User characteristics}

\section{Assumptions, dependencies and constraints}

\subsection{Domain assumptions}

\begin{enumerate}[label=\textbf{D\arabic* -}]
    \item The user must have a working internet connection.
    \item The user must have provided valid personal information.
    \item The student must be registered to a university.
    \item The university must have provided an organization mail to the student.
\end{enumerate}
