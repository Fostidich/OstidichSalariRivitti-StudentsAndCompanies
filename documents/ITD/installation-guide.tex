\chapter{Installation guide}

Below are the steps to launch the system applications for local hosting.

\section{Install Docker Compose}

The installation guide can be found at the following link.

\begin{center}
\url{https://docs.docker.com/compose/install}
\end{center}

\section{Create the configuration file}

Create, wherever you want, a file named \verb|docker-compose.yml|.

Inside that file write the following text.

\fontsize{7pt}{7pt}\selectfont
\begin{verbatim}
services:

  db:
    image: mysql:latest
    container_name: sc-mysql-database
    environment:
      MYSQL_ROOT_PASSWORD: "IHjw-9940"
      MYSQL_DATABASE: "students_and_companies"
      MYSQL_USER: "sc_admin"
      MYSQL_PASSWORD: "YVxm-3861"
    ports:
      - "3380:3306"
    healthcheck:
      test: ["CMD", "mysqladmin", "ping", "-h", "db"]
      interval: 1s
      retries: 300

  sc-web-server:
    image: fostidich/sc_web_server:latest
    container_name: sc-web-server
    environment:
        VITE_API_SERVER_URL: "http://localhost:5522"
    ports:
      - "8080:80"

  sc-application-server:
    image: fostidich/sc_application_server:latest
    container_name: sc-application-server
    environment:
      JWT_SECRET: "8xzTCUwj0a1CaRttQBtZ5Dc2g78gLfiP"
      DB_DEFAULT_CONNECTION: "Server=db;Database=students_and_companies;User ID=sc_admin;Password=YVxm-3861;"
      CONNECTION_URL: "http://0.0.0.0:5000"
    ports:
      - "5522:5000"
    depends_on:
      db:
        condition: service_healthy
\end{verbatim}
\normalsize

\section{Command line commands}

Open a terminal and navigate to the folder containing the \verb|docker-compose.yml| file that you've just created.

Depending on how Docker Compose has been installed, run the corresponding command: use

\begin{center}
\verb|docker compose up --pull always|
\end{center}

if you installed Docker Compose as a plugin; use

\begin{center}
\verb|docker-compose up --pull always|
\end{center}

if you installed it as a standalone binary.

\subsection{Notes}

Since Docker Compose can be installed in multiple ways, and since slight differences can be found depending on the running OS, below is a list of possible solutions to try if the previous command fails.

\begin{itemize}

    \item If you're using Docker Desktop, be sure that it has been previously launched.
    \item If on Windows, try to launch the terminal (CMD or PowerShell) with administrator rights.
    \item If on MacOS or Linux, try prepending a \verb|sudo| before the previous commands.
    \item It may happen that the ports that have been set in the \verb|docker-compose.yml| are already in use.
    If so, feel free to change them, but make sure they still match accordingly.

\end{itemize}

\section{Browser page}

Open the browser and go the the link \url{http://localhost:8080}.

\section{Stopping the container}

To stop the containers, use

\begin{center}
\verb|docker compose down| \\
\verb|docker-compose down|
\end{center}

depending on your installation.

\subsection{Resources cleaning}

To free up space and clean up unused Docker resources, you can run the following command, which will delete all unused containers, images, volumes, and networks.

\begin{center}
\verb|docker system prune -a|
\end{center}


