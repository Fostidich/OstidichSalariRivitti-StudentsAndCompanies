\chapter{Test cases}

\section{Testing}
This section outlines the test procedures we conducted to evaluate both the functionality and adherence to the project product's functions, as outlined in the RASD provided.
Each test case was designed to include the most relevant usage scenarios from the user's point of view.
Below is a complete list of the test cases performed to validate product performance and compliance.
For each one, the objective, the steps performed, the cases considered and the results obtained are indicated.

\subsection{SIGN UP AND LOGIN FOR A STUDENT}\label{subsec:sign-up-and-login-for-a-student}
\begin{itemize}
    \item \textbf{Goal:} To verify that the user can sign up and log in.

    \item \textbf{Steps:}
    \begin{enumerate}
        \item Open the application.
        \item Click on the "Sign up" button.
        \item Fill in the required fields.
        \item Click on the "Sign up" button.
        \item  Fill in the required fields.
        \item Click on the "Log in" button.
    \end{enumerate}
    \item \textbf{Cases:}
    \begin{itemize}
        \item The user fills in all the fields correctly.
        \item The user fills in the fields incorrectly.
        \item Already existing email.
        \item Already existing phone number.
        \item Upload a big file as CV.
    \end{itemize}
    \item \textbf{Results:}
    \begin{itemize}
        \item \textbf{Successes:}
        \begin{itemize}
            \item The user can sign up and log in to the application.
        \end{itemize}
        \item \textbf{Issues Found:}
        \begin{itemize}
            \item A user can sign up with credential that already has been used or in any case the app does not give errors if the email or phone number has already been used for another account.
            \item A user can use an invalid format of the email, the feedback of the system shows that the user has been registered successfully but the user can't log in.
        \end{itemize}
    \end{itemize}
\end{itemize}

\subsection{SIGN UP AND LOGIN FOR A COMPANY}\label{subsec:sign-up-and-login-for-a-company}
\begin{itemize}
    \item \textbf{Goal:} To verify that the user can sign up and log in.

    \item \textbf{Steps:}
    \begin{enumerate}
        \item Open the application.
        \item Click on the "Contact us" button.
    \end{enumerate}
    \item \textbf{Cases:}
    \begin{itemize}
        \item The company contact the platform to create an account.
    \end{itemize}
    \item \textbf{Results:}
    \begin{itemize}
        \item \textbf{Issues Found:}
        \begin{itemize}
            \item The company can't contact the platform to create an account. This process can't be tested because the button doesn't work.
        \end{itemize}
    \end{itemize}
\end{itemize}

\subsection{PROFILE AND CV MANAGEMENT}\label{subsec:profile-and-cv-management}
\begin{itemize}
    \item \textbf{Goal:} To verify that the user can upload the CV.

    \item \textbf{Steps:}
    \begin{enumerate}
        \item Open the application.
        \item Click on the "Sign up" button.
        \item Fill in the required fields including the CV.
        \item Click on the "Sign up" button.
    \end{enumerate}
    \item \textbf{Cases:}
    \begin{itemize}
        \item Upload a big file as CV.
    \end{itemize}
    \item \textbf{Results:}
    \item \textbf{Successes:}
    \begin{itemize}
        \item The user can sign up and log in to the application.
    \end{itemize}
    \item \textbf{Issues Found:}
    \begin{itemize}
        \item The application doesn't check the cv is major than 5 MB (In the RASD is written that the file size limit is 5MB R8).
        \item The system can't give the possibility to the user to update the CV in any time after the registration (RASD R11).
    \end{itemize}
\end{itemize}


\subsection{INTERNSHIP MANAGEMENT}\label{subsec:internship-management}
\begin{itemize}
    \item \textbf{Goal:} To verify that the company can create edit and delete an internship.

    \item \textbf{Steps:}
    \begin{enumerate}
        \item Login in the application.

    \end{enumerate}
    \item \textbf{Cases:}
    \begin{itemize}
        \item Create a new internship.
        \item Edit an internship.
        \item Delete an internship.
    \end{itemize}
    \item \textbf{Results:}
    \item \textbf{Successes:}
    \begin{itemize}
        \item The company can create an internship.
    \end{itemize}
    \item \textbf{Issues Found:}
    \begin{itemize}
        \item The company can't edit or delete an internship (RASD R13).
    \end{itemize}
\end{itemize}

\subsection{CV/INTERNSHIP DESCRIPTION SUGGESTION}\label{subsec:cv-internship-description-suggestion}
According to the project specification, this feature is not implemented.


\subsection{INTERNSHIP SEARCH AND RECOMMENDATION}\label{subsec:internship-search-and-recommendation}
\begin{itemize}
    \item \textbf{Goal:} Verify that the system allows companies to create internships, students to search for internships using skills and filters, and students to receive personalized internship recommendations.

    \item \textbf{Steps:}
    \begin{enumerate}
        \item Log in as a \textbf{Company}.
        \item Navigate to the "add internship" section.
        \item Fill in the internship details, including the required skills.
        \item Click the "Submit" button.
        \item Log out and switch to a \textbf{Student} account.
        \item Navigate to the "Application" section.
        \item Use the search bar to enter one of the required skills (e.g., "Python").
        \item Verify that the newly created internship appears in the results.
        \item Apply the "Location" filter with different values (e.g., "Amsterdam", "Remote", "Milan").
        \item Observe whether the internship results change based on the selected location.
        \item Navigate to the "Recommendations" section to view personalized suggestions.
    \end{enumerate}
    \item \textbf{Cases:}
    \begin{itemize}
        \item Correct flow
    \end{itemize}
    \item \textbf{Results:}
    \begin{itemize}
        \item \textbf{Successes:}
        \begin{itemize}
            \item The system correctly allows companies to create internships.
            \item Searching for internships using skills works as expected.
        \end{itemize}
        \item \textbf{Issues Found:}
        \begin{itemize}
            \item The location filter does not work properly; results remain the same regardless of the selected location.
            \item The recommendation system is unstable: initially, it provided suggestions correctly. Later, it caused critical errors, such as the student being unexpectedly switched into a company profile previously accessed, or the application crashing when clicking on the "Recommendations" section.
        \end{itemize}
        \item The feature to save internships to "Favorites" has not been implemented.
    \end{itemize}

\end{itemize}


\subsection{INTERNSHIP APPLICATION PROCESS}\label{subsec:internship-application-process}
\begin{itemize}
    \item \textbf{Goal:} To verify the hiring process.

    \item \textbf{Steps:}
    \begin{enumerate}
        \item Student applies for an internship.
        \item Company sees the application.
    \end{enumerate}
    \item \textbf{Cases:}
    \begin{itemize}
        \item Company sees the application and their CVs.
        \item Company accepts the application.
        \item Company rejects the application.
        \item Student sees the result.
        \item Student apply for more than one internship.
    \end{itemize}
    \item \textbf{Results:}
    \begin{itemize}
        \item \textbf{Issue found:}
        \begin{itemize}
            \item The company can't see the CV uploaded by the student, but only a dummy document (RASD R15).
            \item A student can apply for other internships even if he has already an ongoing internship and then a new company can accept the application so the student can have more than one ongoing internship.
        \end{itemize}
    \end{itemize}
\end{itemize}

\subsection{QUESTIONNAIRE AND INTERVIEW}\label{subsec:questionnaire-and-interview}
\begin{itemize}
    \item \textbf{Goal:} Verify that the system allows companies to add questionnaires and interviews for candidates, and that students can see the added links.

    \item \textbf{Steps:}
    \begin{enumerate}
        \item Log in as a \textbf{Company}.
        \item click on the internship in which you want to add the questionnaire and the interview for the new candidate.
        \item Select a new candidate from the list.
        \item Click on the "Add Questionnaire" button.
        \item Enter the questionnaire details, including a link to the form.
        \item Click "Confirm" and verify that the questionnaire is added to the candidate’s profile.
        \item Click on the "Add Interview" button.
        \item Enter the interview details, including a date, time, and a link to the meeting.
        \item Click "Confirm" and verify that the interview is added to the candidate’s profile.
        \item Log out and switch to a \textbf{Student} account.
        \item Navigate to the "Application" section.
        \item Verify that the Questionnaire link appears in the corresponding section.
        \item Verify that the Interview link appears in the corresponding section.
    \end{enumerate}

    \item \textbf{Test Cases:}
    \begin{itemize}
        \item Correct flow
    \end{itemize}

    \item \textbf{Results:}
    \begin{itemize}
        \item Companies can successfully add questionnaires to candidates.
        \item Companies can successfully add interviews to candidates.
        \item Students can see both the questionnaire and interview links in their application section.
    \end{itemize}
\end{itemize}


\subsection{NOTIFICATIONS AND REMINDERS}\label{subsec:notifications-and-reminders}
\begin{itemize}
    \item \textbf{Goal:} To verify that the user can receive notifications and reminders.

    \item \textbf{Steps:}
    \begin{enumerate}
        \item Login in the application.
        \item Search the button of the notifications.
    \end{enumerate}
    \item \textbf{Cases:}
    \begin{itemize}
        \item The user receives a notification.
        \item The user receives a reminder.
    \end{itemize}
    \item \textbf{Results:}
    The user has no the notification button (RASD R26),
    however there is the recommendation button, but if a student clicks on it, the app crashes.

\end{itemize}

\subsection{COMPLAINTS AND FEEDBACK}\label{subsec:complaints-and-feedback}
\begin{itemize}
    \item \textbf{Goal:} Verify whether the system allows students and companies to submit feedback about internships and whether the feedback is used for optimizing the recommendation algorithm.

    \item \textbf{Steps:}
    \begin{enumerate}
        \item Log in as a \textbf{Student}.
        \item Navigate to the Profile section.
        \item Click on the "Send Feedback" button.
        \item Select a rating.
        \item Write a comment.
        \item Click "Submit".
        \item Repeat the same steps while logged in as a \textbf{Company}.
    \end{enumerate}

    \item \textbf{Test Cases:}
    \begin{itemize}
        \item Correct flow – The user submits feedback with a rating and a comment.
        \item Feedback submission with an empty or incomplete comment field.
        \item Multiple submissions – The user submits feedback multiple times.
    \end{itemize}

    \item \textbf{Results:}
    \begin{itemize}
        \item \textbf{Successes:}
        \begin{itemize}
            \item The system allows students and companies to submit feedback correctly.
            \item Feedback is only submitted when both a rating and a comment are provided.
        \end{itemize}
        \item \textbf{Issues Found:}
        \begin{itemize}
            \item Users can submit feedback multiple times without restriction.
            \item There is no way to view submitted feedback after sending it.
        \end{itemize}
        \item \textbf{Not Implemented:}
        \begin{itemize}
            \item \textbf{R27:} the system does not use feedback for internships ended.
            \item \textbf{R28:} According to the project specification, this feature is not implemented.
            \item \textbf{R29:} According to the project specification, this feature is not implemented.
        \end{itemize}
    \end{itemize}
\end{itemize}


\subsection{CHATROOM}\label{subsec:chatroom}
According to the project specification, this feature is not implemented.


\section{Further notes}\label{sec:further-notes}
We would have advised the development team to create a website instead of an app (for android only) as it would have been much more to implement. Furthermore, the app does not work properly as it crashes if the ‘recommendation’ button is clicked, the company can only download a dummy document instead of the real CV uploaded by the student.
However, the prototype provided to us does not seem to implement all the functions in the project specification correctly, it is not user-friendly.
