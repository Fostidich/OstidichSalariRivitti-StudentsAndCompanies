\chapter{Architectural design}

\section{Overview}

In the S\&C application, four high-level components can be distinguished: the backend server, the DBMS, the email service, and the frontend interface.

The backend server is the main component of the system, as it contains every piece of business logic required to satisfy the goals specified in the RASD document.
The role of the DBMS is the usual, which is providing interaction with a database by executing command operations.
The email service is used to validate the user account at sign up time, since the verification link is sent via email.  
Lastly, the frontend interface is the web page reachable by the user from a browser.

\section{Components view}

\htitle{Frontend interface}
This is the the web application used by the client, which is reachable by any browser.
The full range of operations that it provides can only be used after the authentication process, performed by the \textit{authentication service}, results successful.
It allows both students and companies to perform a certain set of actions based on the type of user.

\htitle{Web application}
This is the intermediary between the frontend and the backend.
Its purpose is to manage the browser UI pages independently and transparently from the point of view of the other server components.
Its APIs are meant to ease the HTML body filling process done from other components.

\htitle{Authentication service}
This component provides the set of procedures required to handle the authentication of a user into the system.
It manages the log in and sign up phases.

\htitle{Email service}
The main goal of this component is to implement all the procedures that the server requires in order to send email messages to the user.
Mainly, this component allows the \textit{authentication service} to send a verification email when handling the sign up of a user.  

\htitle{Notification service}
The purpose of this component is to collect and manage all the notifications that a user receives.
It defines the notification types and allows other components to easily send them.

\htitle{Query service}
This service acts as a mediator between the server components and the DBMS.
It uses the DBMS APIs to implement a set of functions which have the sole purpose of manipulating the database or retrieving information from it.

\htitle{Recommendation service}
This service provides the algorithms necessary for finding suitable advertisements and candidates.
These are called when the user opens the feed in the home page, or when looking for strong matches that are to be notified. 

\htitle{Suggestion service}
This service provides the algorithms necessary for evaluating the user profiles, in order to propose enhancements.
These are called by the \textit{profile manager} component when the user provides new profile information. 

\htitle{Profile manager}
Users must be able to insert and edit their profile information.
Validity checks and DB updates are performed by this component.
The \textit{query service} and the \textit{suggestion service} APIs are used when data is to be updated.

\htitle{Enrollment manager}
This component manages all the selection process phases, from the student application request to the start of the internship.
It also handles the questionnaires that companies send to students to fill in.

\htitle{Internship manager}
This component handles the operations that the internship may allow or require when it is in the ongoing status.
Along initializing and interrupting the internship, it permits users to visualize its annex information.

\htitle{Complaint manager}
This component allows students and companies to send complaints regarding an ongoing internship, hence allowing the university to visualize and handle them.
Since complaints must have a valid reference to an ongoing internship, the \textit{internship manager} component APIs are here used.

\htitle{Feedback manager}
When an internship ends, both parties are requested to fill in a feedback form.
This component handles this operation, by sending, receiving and acting accordingly to the feedback forms content. 
Since feedback forms must have a valid reference to an internship that has concluded, the \textit{internship manager} component APIs are here used.

\section{Runtime view}
\section{Component interfaces}
\section{Selected architectural styles and patterns}
\section{Other design decisions}
