\chapter{Introduction}

\section{Purpose}

This document contains the design description of the Students\&Companies system.
It includes the architectural design, the user interface design, and the descriptions of all the operations that the system will perform.
It also shows how the requirements and the use cases detailed in the RASD document are satisfied by the design of the system.
This document is intended to be read by the developers, the testers and the project managers of the system.
It is also intended to be used as a reference for any future maintenance.

\section{Scope}

Students\&Companies (S\&C) is a platform designed to connect university students with companies offering internships. 
It simplifies the internship searches of students and the projects advertisement for companies.
The platform employs recommendation mechanisms to match students and companies based on experience, skills, and project requirements. 
S\&C also supports the selection process by managing interviews and collecting feedbacks. 
Additionally, it provides suggestions for improving CVs and project descriptions.

A more detailed description of the system can be found in the RASD document.
This document provides a detailed description of the design that is to implement the requirements and the use cases found in the RASD document.

\section{Definitions, acronyms, abbreviations}

\begin{itemize}
    \item \textbf{Internship project}: the description of the skills, technologies and roles the student will be working with during the internship, along with the set of tasks that will be covered
    \item \textbf{Internship advertisement}: the public post created by companies to promote available internships on the platform, aimed at attracting suitable candidates by highlighting its key aspects
    \item \textbf{Internship information}: general data about the (ongoing) internship, including the elapsed and remaining time, the compensation and the description of the project the student is working on
    \item \textbf{Enrollment request}: the submission of a student to indicate interest in a specific internship, initiating the selection process by formally applying
    \item \textbf{Enrollment suggestion}: the recommendation made by the platform to guide students in finding projects that best suit them
    \item \textbf{Custom questionnaire}: the tailored set of questions used by companies during interviews to assess a candidate fit for the internship
    \item \textbf{Candidate student}: a student who has applied for an internship and is currently under consideration by a company, moving forward in the selection process
    \item \textbf{Eligible student}: a student who meets the qualifications for an internship, making them viable candidates for recommendation and application
    \item \textbf{Suitable student}: a student who meets the qualifications for an internship, making them potential candidates to be recommended in the companies feed
    \item \textbf{Complaint}: a report submitted by a student or company to the university, regarding issues during the internship, such as unmet expectations, mistreatments, or procedural problems
    \item \textbf{Feedback form}: a structured form for students and companies used to provide feedback on their internship experience, enabling the platform to gather data for analysis, improvements, and recommendations
\end{itemize}

\subsection{Acronyms}

\begin{itemize}
    \item \textbf{S\&C}: Students\&Companies
\end{itemize}

\subsection{Abbreviations}

\begin{itemize}
    \item \textbf{Rn}: n-th requirement
    \item \textbf{RVn}: n-th runtime view
\end{itemize}

\section{Revision history}

\begin{itemize}
    \item \textbf{Revised on}: \today
    \item \textbf{Version}: 1.0
    \item \textbf{Description}: document initial release
\end{itemize}

\section{Reference documents}

\begin{itemize}
    \item \textbf{Polimi Software Engineering 2 AY 2024/2025 assignment document}: goal, schedule and rules of the requirement engineering and design project
    \item \textbf{Polimi Software Engineering 2 AY 2024/2025 course slides}: the lecture slides provided during the course
\end{itemize}

\section{Document structure}

\begin{itemize}
    \item \textbf{Chapter 1}: this section provides a brief description of the purpose and the scope of the system; moreover it contains the definitions, acronyms and abbreviations used in the document.
    \item \textbf{Chapter 2}: this section provides a description of the architecture of the system, including the components and the interfaces between them; it also includes the runtime view of the most important operations of the system, along the deployment view and the architectural styles and patterns used.
    \item \textbf{Chapter 3}: here are included the mockups of the user interfaces.
    \item \textbf{Chapter 4}: this section shows how the requirements described in the RASD document are satisfied by the design implementation.
    \item \textbf{Chapter 5}: here is included a step-by-step plan for the implementation and testing of the system.
    \item \textbf{Chapter 6}: this section highlights the effort spent to redact this document by each member of the group.
\end{itemize}

